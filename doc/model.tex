%%%%%%%%%%%%%%%%%%%%%%%%%%%%%%%%%%%%%%%%%%%%%%%%%%%%%%%%%%%%%%%%%%%%%
% LaTeX Template: Project Titlepage Modified (v 0.1) by rcx
%
% Original Source: http://www.howtotex.com
% Date: February 2014
% 
% This is a title page template which be used for articles & reports.
% 
% This is the modified version of the original Latex template from
% aforementioned website.
% 
%%%%%%%%%%%%%%%%%%%%%%%%%%%%%%%%%%%%%%%%%%%%%%%%%%%%%%%%%%%%%%%%%%%%%%

\documentclass[12pt]{report}
\usepackage[a4paper]{geometry}
\usepackage[myheadings]{fullpage}
\usepackage{fancyhdr}
\usepackage{lastpage}
\usepackage{graphicx, wrapfig, subcaption, setspace, booktabs}
\usepackage[T1]{fontenc}
\usepackage[font=small, labelfont=bf]{caption}
\usepackage{fourier}
\usepackage[protrusion=true, expansion=true]{microtype}
\usepackage[english]{babel}
\usepackage{sectsty}
\usepackage{url, lipsum}
\usepackage[inline]{enumitem}
\usepackage{amsmath}
\usepackage{setspace}
\usepackage{hyperref}
\usepackage{amsmath}
\usepackage{algorithm,algpseudocode}
\newcommand{\vars}{\texttt}
\newcommand{\func}{\textrm}
\let\oldReturn\Return
\renewcommand{\Return}{\State\oldReturn}
\doublespacing
\renewcommand{\arraystretch}{0.8}

\newcommand{\HRule}[1]{\rule{\linewidth}{#1}}
\onehalfspacing
\setcounter{tocdepth}{5}
\setcounter{secnumdepth}{5}

%-------------------------------------------------------------------------------
% HEADER & FOOTER
%-------------------------------------------------------------------------------
\pagestyle{fancy}
\fancyhf{}
\setlength\headheight{15pt}
\fancyhead[L]{Design Practices}
\fancyhead[R]{IIT DELHI}
\fancyfoot[R]{Page \thepage\ of \pageref{LastPage}}
%-------------------------------------------------------------------------------
% TITLE PAGE
%-------------------------------------------------------------------------------

\begin{document}

\title{ \normalfont  \textsc{Department of Computer Science and Engineering} \\[2pt]
		\normalfont  {IIT Delhi}
		\\ [2.0cm]
		\HRule{0.5pt} \\
		\LARGE \textbf{\uppercase{Mathematical Model of a Flocking Simulation Software}} \\[2pt]
        \large \textbf{\uppercase{(For the Starlings Murmuration)}}
		\HRule{2pt} \\ [0.5cm]
		\normalsize \date{7 May, 2018} \vspace*{5\baselineskip}}

\date{}

\author{
		Divyanshu Saxena | Akshat Khare \\ 
		Design Practices | COP 290   \\
		 }

\maketitle
\tableofcontents
\newpage

%-------------------------------------------------------------------------------
% Section title formatting
\sectionfont{\scshape}
%-------------------------------------------------------------------------------

%-------------------------------------------------------------------------------
% BODY
%-------------------------------------------------------------------------------

\section*{Introduction}
\addcontentsline{toc}{section}{Introduction}
% \lipsum
We have to design and implement a software for Boid Flocking Simulation. Boids is an artificial life program, developed by Craig Reynolds in 1986, which simulates the flocking behaviour of birds. In our particular case, we are aiming to develop the simulation for the starlings murmuration.
The software should have the following functionalities:
\begin{enumerate}
\item It should be able to approximate the flocking of starlings with some rules, that will form the model.
\item Given the model, we will computationally simulate the phenomenon by modeling each bird as an independent agent communicating and cooperating with other neighboring agents, adhering to the set of rules defined in our model.
\item It should be possible to measure/quantify the software simulation from a realistic flocking. This may make the use of average energy spend by each bird, the angular momentum and the force that each bird has to withstand in a typical flight ritual.
\end{enumerate}

The remaining document contains the rules formulated for the model, mathematical model of the rules and other algorithmic optimisations.

\newpage
\section*{Flocking Rules}
\addcontentsline{toc}{section}{Flocking Rules}
The three most basic rules for simulation are as follows:
\begin{enumerate}
\item Alignment ($d_{align}$)
\item Cohesion ($d_{cohese}$)
\item Separation ($d_{separate}$)
\end{enumerate}
The above rules define the motion of a single boid with reference to the entire flock.
For every boid, the effect of other flock members can be formulated using the above rules and the parameters as mentioned in the corresponding parenthesis.  
\newline
Each parameter denotes the distance, within which the boid-boid interaction shall be activev. Typically, the order for the three distances is:  $d_{separate} < d_{align} < d_{cohese}$.
\newline
Besides these, the following rules must also be implemented so as to bring about the simulated behaviour of the boids in presence of external factors:
\begin{enumerate}
\item Obstacle Avoidance
\item Boundary Restriction
\item Energy Considerations
\end{enumerate}


\subsection*{Alignment}
\addcontentsline{toc}{subsection}{Alignment}
Each boid copies the direction and speed of any other boid which is within the velocity matching range  
$d_{align}$. This rule causes groups of boids to move towards a similar general direction. Each boid steers towards the average heading of flockmates, in the neighbouring area within the radius of $d_{align}$ from the present character.
\newline
The influence of the movement of a boid on another boid decreases with increasing distance between them. 

\subsection*{Cohesion}
\addcontentsline{toc}{subsection}{Cohesion}
Each boid moves towards the perceived center of gravity of all neighboring boids, present within the centering range $d_{cohese}$. This rule causes localized flocking to occur, so that little internal order occurs over time. Each boid steers to move toward the average position of flockmates, in the area within the radius of $d_{cohese}$, with respect to the boid. 

\subsection*{Separation}
\addcontentsline{toc}{subsection}{Separation}
Each boid avoids any other boid which is within the collision avoid range $d_{separate}$ This rule restricts the boids from overlapping visually, as it causes them to actively move away from each other when nearby. Each boid steers to avoid crowding flockmates, in the neighbouring area within the radius of $d_{separate}$, with respect to the boid. 
\newline
The boid tries to move away from other boids that are closer to it. Hence, separation is inversely proportional to the distance between two boids.

\subsection*{Obstacle Avoidance}
\addcontentsline{toc}{subsection}{Obstacle Avoidance}
Each boid avoids any obstacle which is within the collision avoid range $d_{avoid}$. This rule causes them to actively move away from the obstacle when nearby.  
\newline  
The boid tries to move away from the obstacle. The driving force for the boid to move away from the obstacle gets larger with decreasing distance between the boid and the obstacle. Hence, this is inversely proportional to the distance between two boids.

\subsection*{Boundary Restriction}
\addcontentsline{toc}{subsection}{Boundary Restriction}
Each boid restricts its motion within the boundary range $d_{boundary}$, with respect to a common reference point. This rule restricts the boids from drifting too far away and causes them to revolve around a common centre point.
\newline  
The boids can be thought to be moving freely when they are well within the boundary. However, as the boids get closer to the boundary, the force to get back to the centre point increases and ultimately, when the boid reaches the boundary, it turns back and moves within the boundary range.

% End of specifications
\newpage
\section*{Mathematical Model and Pseudocode}
\addcontentsline{toc}{section}{Mathematical Model and Pseudocode}
Based on our Flocking Rules, we develop the corresponding mathematical model for any boid in our sample space. Every Boid can be thought of a point mass that experiences some force, which eventually drives it. This force is a weighted resultant of all the forces experienced by the point mass, as it obeys the flocking rules specified earlier. Due to this force, the boids change their velocity so as to move to new positions, which are needed to be calculated.
\newline
For each of the rules, we can calculate the change in its position, at every instant, that the boid must make in order to follow the rules. This move is added to the current position of the boid to get the new position.
\newline
Below, we have given the mathematical specification to calculate this momentary change in position due to each of the specified Flocking Rules.
\newline\newline
Each of the following algorithms make use of an array of boids, \(friends[]\), for each Boid under consideration. The \(friends[]\) array consists of the boids that are within a radius of $d_{cohese}$ from the current boid.

\subsection*{Alignment}
\addcontentsline{toc}{subsection}{Alignment}
% Alignment
\begin{algorithm}
\caption{Alignment}\label{alignment}
\begin{algorithmic}[1]
\Function{getAlignment}{}
	\State $\vars{steer} \gets ZERO\_VECTOR$
	\State $\vars{count} \gets 0$
	\For{\texttt{each boid b in friends[]}}
    	\State $\vars{d} \gets \texttt{distance between current boid and b}$
        \If{$\vars{d} > 0$ AND $\vars{d} < d_{align} $}
            \State $\vars{steer} \gets \vars{steer} + \vars{b.move}/\vars{d} $
            \State $\vars{count} \gets \vars{count} + 1 $ 
		\EndIf
	\EndFor
    \If{$\vars{count} > 0$}
        \State $\vars{steer} = \vars{steer}/\vars{count}$
    \EndIf
    \Return $\vars{steer}$
\EndFunction
\end{algorithmic}
\end{algorithm}
In the line 7 above, the influence of the boid b is inversely proportional to its distance from the current boid, as established in the Rules section before. 

\subsection*{Cohesion}
\addcontentsline{toc}{subsection}{Cohesion}
% Alignment
\begin{algorithm}
\caption{Cohesion}\label{cohesion}
\begin{algorithmic}[1]
\Function{getCohesion}{}
	\State $\vars{steer} \gets ZERO\_VECTOR$
	\State $\vars{sum} \gets ZERO\_VECTOR$
	\State $\vars{count} \gets 0$
	\For{\texttt{each boid b in friends[]}}
        \State $\vars{sum} \gets \vars{sum} + \vars{b.position} $
        \State $\vars{count} \gets \vars{count} + 1 $ 
	\EndFor
    \If{$\vars{count} > 0$}
        \State $\vars{steer} = \vars{sum}/\vars{count}- \vars{position}$
    \EndIf
    \Return $\vars{steer}$
\EndFunction
\end{algorithmic}
\end{algorithm}
The function above finds the centre of gravity of all the boids in the \(friends[]\) array and the boid tries to move towards this centre of gravity. Farther the centre, more is the force experienced by the boid. 

\subsection*{Separation}
\addcontentsline{toc}{subsection}{Separation}
% Alignment
\begin{algorithm}
\caption{Separation}\label{separation}
\begin{algorithmic}[1]
\Function{getSeparation}{}
	\State $\vars{steer} \gets ZERO\_VECTOR$
	\For{\texttt{each boid b in friends[]}}
    	\State $\vars{d} \gets \texttt{distance between current boid and b}$
		\If{$\vars{d} > 0$ AND $\vars{d} < d_{separate}$}
        \State $\vars{diff} \gets \vars{position}-\vars{b.position}$
        \State $\vars{steer} \gets \vars{steer}+\vars{normalized\_diff}\/d $
        \EndIf
	\EndFor
    \Return $\vars{steer}$
\EndFunction
\end{algorithmic}
\end{algorithm}
The function above drives the boid away from any other boid at a distance less than $ d_{separate} $, from the boid. The subtraction is done so as to ensure tht  the boids are drawn apart. The vector away from the boid is divided by the distance between the two boids to make the separation coefficient inversely proportional to the distance.


\subsection*{Obstacle Avoidance}
\addcontentsline{toc}{subsection}{Obstacle Avoidance}
% Alignment
\begin{algorithm}
\caption{Obstacle Avoidance}\label{obstacle}
\begin{algorithmic}[1]
\Function{avoidAvoids}{}
	\State $\vars{steer} \gets ZERO\_VECTOR$
	\For{\texttt{each obstacle a in avoids[]}}
    	\State $\vars{d} \gets \texttt{distance between current boid and a}$
		\If{$\vars{d} > 0$ AND $\vars{d} < d_{avoid}$}
        \State $\vars{diff} \gets \vars{position}-\vars{a.position}$
        \State $\vars{steer} \gets \vars{steer}+\vars{normalized\_diff}\/d $
        \EndIf
	\EndFor
    \Return $\vars{steer}$
\EndFunction
\end{algorithmic}
\end{algorithm}
The above function is exactly similar to the getSeparation pseudocode defined earlier, with the differences in the avoid radius and the $avoids[]$ array.

\newpage
\section*{Further Considerations}
\addcontentsline{toc}{section}{Further Considerations}
Apart from the above specified rules, the following experiments were also made in the current project, to check the correctness of the simulation. The obtained results are specified here:

\subsection*{Performance Improvement}
\addcontentsline{toc}{subsection}{Performance Improvement}
The  first major concern was to improve the performance of the system, by adding more boids. As per a biological paper, a single flock of starlings constitutes of thousands of birds. Hence, for performance improvement, it was necessary to limit the number of boids that should influence a given boid. 
\newline
Next step was to try multi-threading. The software used for the current project is Processing3D. The software, inherently uses threads to separate the computation process from the computation part. However, upon adding threads in the computation part as well (to improve the performance), it was noticed that the process got slower because then, the rendering started waiting for the thread to finish. Hence, implementing user-defined threading wasn't very helpful

\subsection*{Energy Considerations}
\addcontentsline{toc}{subsection}{Energy Considerations}
To get a sense of correctness of the simulation, it was necessary to get a measure to quantify the movement of the boids. The energy of a boid was found to be a good measure to do so. The boid, under the influence of gravity must try to move in a direction perpendicular to the force of gravity. 
\newline
For this, the boids must lose energy when moving upwards and gain some energy when moving downwards. This loss in energy is implemented by the loss in velocity (or move) of the boid. This way, the effect of energy shall be more pronounced.

\newpage
\section*{References}
\addcontentsline{toc}{section}{References}
\begin{enumerate}
\item  \textbf{Flocks, Herd and Schools: A Distributed Behaviour} \\
Boids - Background and Update by Craig Reynolds

\item  \textbf{Boids: Pseudocode} \\
Boids Pseudocode by Conrad Parker. \emph{kfish.org}
\end{enumerate}



\end{document}
